\section{Solution implementation}
Here we discuss the implementation of our solution. As already anticipated,
the core of the system is a microservice architecture that communicates
through the REST API. Since every microservice is independent of each
other, the technologies that are used internally in the various modules
do not have to be the same ones. So, apart from keeping a standardized REST
interface among the components, we could develop the microservices with
different programming languages and frameworks. Here are described the
specific implementations of the single services.

\subsubsection{Microservices}


\subsubsection{User auth}

This module was built using Typescript that runs on a Node.js environment after
being compiled into Javascript. The corresponding database was built using MongoDB
that provides a NoSQL structure. The module uses the Express.js library to expose
its REST API and the Mongoose library to communicate with the database. The POST http
method was used to perform operations that edit entries in the database, such as
during the registration. The GET http method, instead, was used for operations that
do not modify content in the database, but only read information, such as the token
validation.\\

Here is shown the exposed API:\\

%\emph{http://user_auth_api:3000/[query]}\\

\begin{lstlisting}[language=bash,caption={User auth exposed API}]
    POST register
    JSON body data
        name: string
        surname: string
        email: string
        password: string
    JSON response
        token: string OR error: string

        
    POST login
    JSON body data
        email: string
        password: string OR token: string
    JSON response
        token: string OR error: string

    POST logout
    JSON body data
        email: string
        token: nullable string
    JSON response
        token: string OR error: string

        
    GET validate
    JSON body data
        email: string
        token: string
    JSON response
        token: string OR error: string
\end{lstlisting}


\subsubsection{Restaurant auth}

This module was implemented very similarly to the user auth one.\\

Here is shown the exposed API:\\

%\emph{http://restaurant_auth_api:3000/[query]}\\

\begin{lstlisting}[language=bash,caption={Restaurant auth exposed API}]
    POST register
    JSON body data
        name: string
        email: string
        password: string
    JSON response
        token: string OR error: string
        
    POST login
    JSON body data
        email: string
        password: string OR token: string
    JSON response
        token: string OR error: string


    POST logout
    JSON body data
        email: string
        token: nullable string
    JSON response
        token: string OR error: string


    GET validate
    JSON body data
        email: string
        token: string
    JSON response
        token: string OR error: string
\end{lstlisting}

\subsubsection{Restaurant data}


\subsubsection{Booking}

